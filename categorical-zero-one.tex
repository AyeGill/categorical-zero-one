\documentclass[11pt]{article}
\usepackage{eigilscmds}
\usepackage{tikzit}
\usepackage{geometry}
\usepackage{enumerate}
\input{comonoids.tikzdefs}
\input{comonoids.tikzstyles}

% draft packages
\usepackage{todonotes}
\usepackage{showkeys}

\author{Tobias Fritz and Eigil Fjeldgren Rischel}
\title{The zero--one laws of Kolmogorov and Hewitt--Savage in categorical probability}
\date{\today}

\renewcommand{\sf}{\mathsf}
\DeclareMathOperator{\del}{del}
\begin{document}
\maketitle

\begin{abstract}
	We state and prove the zero--one laws of Kolmogorov and Hewitt--Savage within the setting of \emph{Markov categories}, a category-theoretic approach to the foundations of probability and statistics. This gives general versions of these results which can be instantiated not only in measure-theoretic probability, where they specialize to the standard ones, but also in other contexts such as categories of closed relations between topological spaces, 
	\improve{T: this abstract states what we can hopefully achieve. To be adjusted}
\end{abstract}

\tableofcontents

\section{Introduction}

This is a treatment of the zero--one laws of Kolmogorov and of Hewitt--Savage in the setting of Markov categories.

\paragraph*{Notation.} 

Throughout, $\cC$ is a symmetric monoidal category. To simplify notation, we generally omit mention of the structure isomorphisms by assuming that $\cC$ is strict without loss of generality. We make frequent use of string diagram notation, and in doing so we omit object labels whenever these are obvious from the context.

In \Cref{infprod_semicartesian}, $\cC$ denotes more concretely a semicartesian strict symmetric monoidal category. In \Cref{infprod_markov} and after, $\cC$ denotes a Markov category in the sense of \Cref{markov_cat}. $J$ is a set used for indexing products of objects; the definitions and results of this paper are nontrivial only when $J$ is infinite. $F$ denotes either an arbitrary finite set or more concretely a finite subset of $J$.

\section{Background on Markov categories}

A symmetric monoidal category $\cC$ is \emph{semicartesian} if the monoidal unit $I \in \cC$ is terminal. Equivalently, it is a monoidal category equipped with morphisms
\[
	X \otimes Y \longrightarrow X, \qquad X \otimes Y \longrightarrow Y,
\]
which are natural in $X$ and $Y$ and coincide with the monoidal structure isomorphisms whenever $X = I$ or $Y = I$. Here are the main examples that we will be considering.

\begin{example}
	\label{finstoch}
	$\FinStoch$ is the category of finite sets with \emph{stochastic matrices}, which means that $f : X \to Y$ is a matrix $(f_{xy})_{x \in X,y \in Y}$ of nonnegative real numbers such that $\sum_y f_{xy} = 1$ for every $x$. Stochastic matrices compose by matrix multiplication, which in this context is also known as the \emph{Chapman--Kolmogorov equation}. We consider $\FinStoch$ as a symmetric monoidal category with respect to the cartesian product of sets on objects, and as one would expect the Kronecker product of stochastic matrices on morphisms.
\end{example}

\begin{example}
	\label{cring}
	Let $\CRing_+$ be the category of commutative rings with maps which are merely additive and unit-preserving, considered as a symmetric monoidal category in the obvious way with respect to the tensor product of rings. Then since $\CRing_+$ has the monoidal unit $\Z$ as its initial object, we can conclude that $\CRing_+\op$ is semicartesian monoidal.
\end{example}

\begin{example}
	\label{stoch}
	$\Stoch$ is the category of measurable spaces $(X,\Sigma_X)$ as objects and Markov kernels as morphisms. We sketch the definition here and refer to~\cite[Section~4]{markov_cats} and references therein for the details. A \emph{Markov kernel} $(X,\Sigma_X) \to (Y,\Sigma_Y)$ is a function
	\[
		f \: : \: \Sigma_Y \times X \longrightarrow [0,1], \quad (S,x) \longmapsto f(S|x)
	\]
	assigning to every $x \in X$ a probability measure $f(-|x) : \Sigma_Y \to [0,1]$ in such a way that for every $S \in \Sigma_Y$, the function $x \mapsto f(S|x)$ is measurable. Intuitively, $f$ assigns to every $x \in X$ a random element of $Y$. Composition is again defined by a variant of the Chapman--Kolmogorov equation, and the symmetric monoidal structure is defined in terms of the usual product of measurable spaces. The monoidal unit $I$ is given by any one-element set with its unique $\sigma$-algebra. As a special case, morphisms $I \to (X,\Sigma_X)$ can be identified with probability measures $\sigma_X \to [0,1]$.

	There is a well-known subclass of particularly well-behaved measurable spaces, the \emph{standard Borel spaces} or equivalently Polish spaces equipped with their Borel $\sigma$-algebras. We write $\BorelStoch \subseteq \Stoch$ for the full subcategory of standard Borel spaces with Markov kernels. Since finite products of standard Borel spaces are again standard Borel, $\BorelStoch$ is again semicartesian symmetric monoidal.
\end{example}

Returning to the general theory, in string diagrams the unique morphism $\discard{X} : X \to I$ for an object $X \in \cC$ is denoted by
\[
	\tikzfig{terminal}
\]
The projection maps $\id \otimes \discard{Y} : X \otimes Y \to X$ and $X \otimes Y \to Y$ are correspondingly written as
\[
	\tikzfig{marginals}
\]

Our goal is to use semicartesian monoidal categories such as the above in order to develop aspects of probability theory in categorical terms. As it turns out, doing so requires a bit more structure, in a form which has been axiomatized first by Cho and Jacobs~\cite{cho_jacobs} as \emph{affine CD-categories}, although similar definitions occur in earlier work of Golubtsov~\cite{golubtsov}. We here follow the terminology of our own~\cite[Definition~2.1]{markov_cats}. 

\begin{definition}
	A \emph{Markov category} $\cC$ is a semicartesian symmetric monoidal category where every object $X \in \cC$ is equipped with a morphism
	\begin{equation}
		\label{comultiplication}
		\tikzfig{comultiplication}
	\end{equation}
	which, together with the unique $X \to I$, makes $X$ into a commutative comonoid, and such that
	\begin{equation}
		\label{multiplicativity}
		\tikzfig{multiplicativity}
	\end{equation}
	\label{markov_cat}
\end{definition}

There is a strictification result~\cite[Theorem~10.16]{markov_cats} which guarantees that $\cC$ can be assumed to be strict monoidal, as we do throughout.

A closely related definition was also used in earlier work of Golubtsov~\cite{golubtsov}, who had proposed a closely related definition, and Cho and Jacobs~\cite{cho_jacobs}, where an equivalent definition was used under the term \emph{affine CD-category}.

We think of the comultiplication~\Cref{comultiplication} as a copying operation. The coassociativity and counitality conditions guarantee that copying with any number of output wires is well-defined, and we draw it likewise as a single black dot with any number of outgoing wires, as in
\[
	\tikzfig{triple_copy}
\]

\Cref{finstoch,cring,stoch} are all Markov categories in a canonical way. In $\FinStoch$ and $\Stoch$, the comultiplication morphisms are given indeed by copying, i.e.~by those stochastic matrices or Markov kernels which map an $x \in X$ to the Dirac delta measure at $(x,x) \in X \times X$. We refer to~\cite[Example~2.5 and Section~4]{markov_cats} for the technical details. In $\CRing_+\op$, we define the comultiplication of an object $R \in \CRing_+\op$ to be represented by the multiplication map $R \otimes R \to R$. The multiplicativity condition~\eqref{multiplicativity} then amounts to the fact that the multiplication on a tensor product of commutative rings $R \otimes S$ is given by the defining equation
\[
	(r_1 \otimes s_1) (r_2 \otimes s_2) \, = \, r_1 r_2 \otimes s_1 s_2.
\]

\begin{definition}[{\cite[Definition~10.1]{markov_cats}}]
	Let $\cC$ be a Markov category. A morphism $f : X \to Y$ in $\cC$ is \emph{deterministic} if it is a comonoid homomorphism,
	\[
		\tikzfig{multiplication_natural}	
	\]
\end{definition}

As per~\cite[Remark~10.12]{markov_cats}, the deterministic morphisms form a symmetric monoidal cartesian subcategory $\cC_\detc \subseteq \cC$ which contains all structure morphisms of $\cC$, including the comultiplications themselves. The fact that all morphisms are comonoid homomorphisms implies that $\cC_\detc$ is cartesian monoidal.

In $\FinStoch$, the deterministic morphisms $X \to Y$ are exactly the $\{0,1\}$-valued stochastic matrices. Since these can be identified with honest functions $X \to Y$, we conclude that $\FinStoch_\detc$ is equivalent to the category of finite sets. In $\Stoch$, the deterministic morphisms $f : X \to Y$ are exactly those Markov kernels for which $f(S|x) \in \{0,1\}$ for every $S \in \Sigma_Y$ and $x \in X$. In other words, as the term suggests, the deterministic morphisms are those which do not involve any randomness. In $\CRing_+\op$, the deterministic morphisms $R \to S$ are precisely those morphisms which are represented by additive unital maps $S \to R$, i.e.~the ring homomorphisms. Thus the subcategory of deterministic morphisms is exactly the opposite of the usual category of commutative rings.

\section{Infinite tensor products in semicartesian monoidal categories}
\label{infprod_semicartesian}

\todo[inline]{Let's simplify notation by writing $X_J$ and $X_F$ instead of writing $\bigotimes$ every time, as you've already partly done}

Zero--one laws involve infinitely many random variables. In the setting of Markov categories, this means that we need to consider morphisms whose codomain is an ``infinite tensor product'' in a suitable sense. We start in this section by considering infinite tensor products in semicartesian symmetric monoidal categories.

Suppose that $(X_i)_{i \in J}$ is any family of objects, where the indexing set $J$ is typically infinite. For any finite subset $F \subseteq J$, we also write $X_F := \bigotimes_{i \in F} X_i$ for simplicity of notation. By Kolmogorov's extension theorem, the joint distribution of infinitely many random variables is uniquely determined by the family of distributions of all finite subsets of these variables. This motivates our general definition of the infinite tensor product: if $F \subseteq F' \subseteq J$ are two finite subsets, then the fact that $\cC$ is semicartesian monoidal gives us \emph{marginalization morphisms}
\[
	\pi_{F',F} \: : \: X_{F'} \longrightarrow X_F.
\]
Via these maps, the finite tensor products $(X_F)_{F \subseteq J}$ make up a cofiltered diagram in the form of a functor from the poset of finite subsets of $J$, ordered by reverse inclusion, to $\cC$.

\begin{definition}
	\label{semicartesian_infproduct}
	Let $(X_i)_{i \in J}$ be a family of objects in $\cC$. Then the \emph{infinite tensor product}
	\[
		X_J \: := \: \bigotimes_{i \in J} X_i
	\]
	is the limit of the diagram $F \mapsto X_F$, indexed by the poset of finite subsets $F \subset J$ ordered by reverse inclusion, if this limit exists.
\end{definition}

We will refer to the structure maps $\pi_F : X_J \to X_F$ as \emph{finite marginalizations}. % We also write $\pi_j$ as shorthand for $\pi_{\{j\}}$.

\begin{remark}
	If $J$ is finite, then the infinite tensor product exists and coincides with the standard tensor product $\bigotimes_{i \in J} X_i$. In a general Markov category, the relevant cofiltered limit need not exist for infinite $J$, in which case the infinite tensor product does not exist either (as in \Cref{infprods_finstoch}).
\end{remark}

\begin{example}
	\label{infprods_cring}
	The dual definition of infinite tensor products of algebraic structures as \emph{filtered colimits} of finite tensor products is well-known in the literature, e.g.~in the case of C*-algebras~\cite[p.~315]{blackadar}. Thus in the case where our Markov category is $\CRing_+\op$, we recover the usual folklore definition of an infinite tensor product of rings $\bigotimes_{i \in J} R_i$ in terms of formal sums of elementary tensors, where an elementary tensor is a family of elements $(r_i)_{i \in J}$ such that all but finitely many are equal to the respective unit. Note that we do not yet consider the multiplication on $\bigotimes_{i \in J} R_i$, since \Cref{semicartesian_infproduct} is not yet concerned with the comonoid structures on the objects. We will get to this in the next section.
	
	Intuitively, our definition matches up with these algebraic ones under the categorical duality of algebra and geometry, where our definition is on the geometrical side of the duality.
\end{example}

\begin{remark}
	In those semicartesian monoidal categories $\cC$ that are of interest to us, morphisms $I \to X$ play the role of \emph{probability measures} on $X$. Thus applying the universal property with respect to maps out of $I$ amounts to requiring the \emph{Kolmogorov extension theorem} to hold in $\cC$: probability measures on an infinite product $X_J$ are in bijection with consistent families of probability measures on the finite products $X_F$ for $F \subseteq J$.
\end{remark}

\begin{example}
	\label{infprods_stoch}
	We now consider infinite tensor products in $\Stoch$, where we would like infinite tensor products to be given by the corresponding infinite products of measurable spaces in the usual sense. Since the Kolmogorov extension theorem does not hold for general (even merely countable) products of measurable spaces~\cite{AJ}, this is not the case without further additional assumptions on the measurable spaces involved. Thus we do not know whether $\Stoch$ has infinite tensor products, although we suspect that it does not; but even if it does, they are in general not the ones that one would like to have for the purposes of probability theory.

	However, the situation improves for countable products in $\BorelStoch$, for which the Kolmogorov extension theorem holds~\cite[Theorem~14.35]{klenke}. In other words, if $\left( (X_i, \Sigma_i) \right)_{i \in \bN}$ is a sequence of standard Borel spaces, then the cartesian product $X_\bN = \prod_{i \in \bN} X_i$ carrying the product $\sigma$-algebra $\Sigma_\bN$ satisfies the universal property of an infinite tensor product with respect to maps out of $I$. In the following, we show that this implies the universal property in general.

	Suppose that
	\[
		\Big( g_F : (A,\Sigma_A) \longrightarrow (X_F, \Sigma_F) \Big)_{F \subseteq J \text{ finite}}
	\]
	is a family of Markov kernels satisfying the compatibility condition $\pi_{F',F} \circ g_{F'} = g_F$ for all finite $F \subseteq F' \subseteq J$. Then for every $a \in A$, the probability measures $g_F(-|a) : \Sigma_F \to [0,1]$ are a compatible family to which the Kolmogorov extension theorem in the form~\cite[Theorem~14.35]{klenke} applies, and we obtain a unique probability measure $g_J(-|a) : \Sigma_J \to [0,1]$ which has the $g_F(-|a)$ as its finite marginals. It remains to be shown that for every $S \in \Sigma_J$, the map $a \mapsto g_J(S|a)$ is measurable. Since limits of pointwise convergent sequences of measurable real-valued functions are again measurable, the set of $S$ for which this measurability holds is closed under countable disjoint union, and it is clearly closed under complements. We therefore have a $\lambda$-system. Since the map is measurable by assumption whenever $S$ is a measurable cylinder set, and the cylinder sets form a $\pi$-system, the $\pi$-$\lambda$-theorem implies that $a \mapsto g_J(S|a)$ is measurable for all $S$ in the $\sigma$-algebra generated by the measurable cylinder sets, which is exactly the product $\sigma$-algebra $\Sigma_J$.

	In conclusion, $\BorelStoch$ indeed has countable tensor products. It is plausible that there is is another subcategory of $\Stoch$ strictly larger than $\BorelStoch$ which has all infinite tensor products; for example, one can try to construct such a subcategory by imposing compactness or perfectness conditions~\cite[\S{451}]{fremlin4}, for which there is a version of the Kolmogorov extension theorem~\cite[Corollary~454G]{fremlin4}. However, we have so far not been able to find a suitable condition on Markov kernels which would be closed under composition, due to problems of perfect measures under mixtures~\cite{ramachandran}.
\end{example}

\begin{example}
	\label{infprods_finstoch}
	Infinite tensor products never exist in $\FinStoch$, in the following sense: if the family $(X_i)$ is such that no $X_i$ is empty and infinitely many of them contain at least two elements, then $\bigotimes_i X_i$ does not exist. One way to see this is to use the fact that the hom-sets of $\FinStoch$ are convex sets in the space of matrices, and that composition distributes over these convex combinations, making $\FinStoch$ into a category enriched in convex sets. Since these hom-sets are finite-dimensional, for every $Y$ and $Z$ there is $n \in \bN$ such that among every $n$ morphism $Y \to Z$, one of them can be written as a convex combination of the others. Now suppose that the product $\bigotimes_i X_i$ existed. Then by choosing varying elements of each $X_i$, we can construct uncountably many morphisms $1 \to \bigotimes_i X_i$ whose marginalizations $1 \to X_F$ are all deterministic. Per the above, one of these hypothetical morphisms can be written as a convex combination of finitely many others. By choosing $F \subseteq J$ suitably, we can achieve that the finite marginalizations $1 \to X_F$ are all distinct. Since they are deterministic by construction, and no deterministic morphism in $\FinStoch$ can be written as a convex combination of other deterministic morphisms, we have arrived at a contradiction.
\end{example}

It is natural to require different infinite tensor products to interact well with another. If $X_J$ is an infinite tensor product of a family $(X_i)_{i \in J}$ and we are given an additional object $X_\ast$ for $\ast \not \in J$, then one will want $X_J \otimes X_\ast$ to be the infinite tensor product $X_{J \sqcup \{\ast\}}$ of the original family together with $X_\ast$, with finite marginalization morphisms given by
\[
		\pi_{F \setminus \{\ast\}} \otimes \id \: : \: X_J \otimes X_\ast \longrightarrow X_{F \setminus \{\ast\}} \otimes X_\ast
\]
for $\ast \in F$, and simply
\[
		\pi_F \otimes \discard{X_\ast} : X_J \otimes X_\ast \longrightarrow X_F 
\]
in case $\ast \not \in F$. It is straightforward to check that these morphisms exhibit $X_J \otimes X_\ast$ as the infinite tensor product $X_{J \sqcup \{\ast\}}$ if and only if the functor $- \otimes X_\ast$ preserves the defining cofiltered limit of the infinite tensor product $X_J$. This motivates the following definition.

\begin{definition}
	\label{has_infprods}
	Let $\cC$ be a semicartesian symmetric monoidal category. We say that $\cC$ \emph{has infinite tensor products} if for every family of objects $(X_i)_{i \in J}$ with $|J| < \kappa$, the infinite tensor product $X_J$ exists, and its defining cofiltered limit is preserved by every functor $Y \otimes -$.

	For $\kappa$ a regular cardinal, we say that $\cC$ \emph{has tensor products of size $<\kappa$} if these conditions hold as soon as $|J| < \kappa$. For $\kappa = \aleph_1$, we also simply say that $\cC$ \emph{has countable tensor products}.
\end{definition}

\todo[inline]{We could also make the preservation condition part of the definition of infinite tensor product. Thoughts?}

As per the above, the preservation condition in this definition amounts to the requirement that the canonical comparison morphism
\[
	X_J \otimes X_\ast \longrightarrow X_{J \sqcup \{\ast\}}
\]
must be an isomorphism, as one would expect intuitively from a notion of infinite tensor product. The explicit constructions of \Cref{infprods_cring,infprods_stoch} show that $\CRing_+\op$ has infinite tensor products and that $\BorelStoch$ has countable tensor products, respectively.

\begin{lemma}
	\label{two_infproducts}
	Suppose that $\cC$ has tensor products of size $< \kappa$. Let $J = J_1 \sqcup J_2$ be a disjoint union with $|J| < \kappa$, and $(X_i)_{i \in J}$ a family of objects in $\cC$. Suppose that the infinite tensor products $X_{J_1}$ and $X_{J_2}$ exist. Then the object
	\[
		X_{J_1} \otimes X_{J_2} 
	\]
	is an infinite tensor product $X_J$ with respect to the finite marginalizations morphisms given by, for every finite $F \subseteq J$,
	\[
		\begin{tikzcd}[column sep=huge]
			\rho_F \: : \: X_{J_1} \otimes X_{J_2} \ar{r}{\pi_{F\cap J_1} \otimes \pi_{F\cap J_2}} & X_{F \cap J_1} \otimes X_{F \cap J_2}.
		\end{tikzcd}
	\]
\end{lemma}
\begin{proof}
	For fixed finite $F_1 \subseteq J_1$, the morphisms
	\[
		\id \otimes \pi_{F_2} \: : \: X_{F_1} \otimes X_{J_2} \longrightarrow X_{F_1} \otimes X_{F_2}
	\]
	exhibit $X_{F_1} \otimes X_{J_2}$ as the cofiltered limit of the $X_{F_1} \otimes X_{F_2}$ by the preservation assumption. Similarly, the morphisms
	\[
		\pi_{F_1} \otimes \id \: : \: X_{J_1} \otimes X_{J_2} \longrightarrow X_{F_1} \otimes X_{J_2}
	\]
	exhibit $X_{J_1} \otimes X_{J_2}$ as the cofiltered limit of the $X_{F_1} \otimes X_{J_2}$. The claim follows as a limit of limits is a limit, and it is easy to see that the diagram shapes much up: the poset of finite subsets of $J_1 \sqcup J_2$ is the product of the posets of finite subsets of $J_1$ and $J_2$.
\end{proof}

More generally, we may consider infinite tensor products of infinite tensor products, $\bigotimes_{k \in K} \bigotimes_{i \in J_k} X_{k,i}$, where now $(J_k)_{k \in K}$ is a family of sets and $(X_{k,i})$ a doubly indexed family of objects. One may think that this doubly infinite tensor product should be isomorphic to the single-step infinite tensor product $\bigotimes_{k \in K, \: i \in J_k} X_{k,i}$. And indeed, if for every finite $F \subseteq \prod_{k \in K} J_k$ we choose any finite $G \subseteq K$ such that $(k,i) \in F$ implies $k \in G$, then we have morphisms
\begin{equation}
	\label{double_infproduct}
	\begin{tikzcd}[column sep=huge]
		\rho_F \: : \: \bigotimes_{k \in K} \bigotimes_{i \in J_k} X_{k,i} \ar{r}{\pi_G} & \bigotimes_{k \in G} \bigotimes_{i \in J_k} X_{k,i} \ar{r}{\bigotimes_{k \in G} \pi_{F \cap J_k}} & \bigotimes_{k \in G, \: i \in F \cap J_k} X_{k,i}
	\end{tikzcd}
\end{equation}
where the object on the right can also be written as $\bigotimes_{(k,i) \in F} X_{k,i}$. It is straightforward to see that this $\rho_F$ does not depend on the particular choice of $G$. We now show that these $\rho_F$'s are finite marginalization morphisms which make the doubly infinite tensor product $\bigotimes_{k \in K} \bigotimes_{i \in J_k} X_{k,i}$ into the single infinite tensor product $\bigotimes_{k \in K, \: i \in J_k} X_{k,i}$.

\begin{lemma}
	Suppose that $\cC$ has tensor products of size $< \kappa$, and let $(J_k)_{k \in K}$ be a family of sets with $|K| < \kappa$ and $|J_k| < \kappa$, and $(X_{k,i})_{i \in J_k, k \in K}$ a doubly indexed family of objects. Then the above $\rho_F$ exhibit the infinite tensor product of infinite tensor products $\bigotimes_{k \in K} \bigotimes_{i \in J_k} X_{k,i}$ as the infinite tensor product $\bigotimes_{k \in K, \: i \in J_k} X_{k,i}$.
\end{lemma}
\begin{proof}
	By repeated application of \Cref{two_infproducts}, the second half of~\Cref{double_infproduct} exhibits $\bigotimes_{k \in G} \bigotimes_{i \in J_k} X_{k,i}$ as the infinite product of the $X_{k,i}$ for $k \in G$ and $i \in J_k$. The claim then follows from a straightforward diagram chase.
\end{proof}

\section{Infinite tensor products in Markov categories}
\label{infprod_markov}

If a Markov category $\cC$ has infinite tensor products, then one should expect a compatibility condition between these infinite tensor products and the comonoid structures on the objects. % In the following string diagrams, we draw the object associated to an infinite tensor product as a double wire.

\begin{definition}
    \label{defn_kolmogorov_ext}
    Let $\cC$ be a Markov category and $(X_i)_{i \in J}$ a family of objects. We say that an infinite product $X_J = \bigotimes_{i \in J} X_i$ is a \emph{Kolmogorov product} if the finite marginalization morphisms $\pi_F : X_J \to X_F$ are deterministic, and the universal property is preserved by every functor $Y \otimes -$.
    
    We say that $\cC$ \emph{has Kolmogorov products} if every family of objects has a Kolmogorov product. For $\kappa$ a regular cardinal, we say that $\cC$ has \emph{Kolmogorov products of size $<\kappa$} if families of size $<\kappa$ have a Kolmogorov product. For $\kappa = \aleph_1$, we say that $\cC$ has \emph{countable Kolmogorov products}.
\end{definition}

In $\BorelStoch$, the infinite tensor products constructed in \Cref{infprods_stoch} are Kolmogorov products. In $\CRing_+\op$, the infinite tensor products from \Cref{infprods_cring} are Kolmogorov products as well as long as one equips $\bigotimes_i R_i$ with the tensor product ring structure, since this one is the only one which makes the canonical inclusions $R_i \to \bigotimes_i R_i$ into ring homomorphisms. These examples illustrate an important point: in some Markov categories, such as $\CRing_+\op$, not all isomorphisms are deterministic, and there may even be isomorphic objects which have no deterministic isomorphism~\cite[Remark~10.9]{markov_cats}. In these categories, whether a given object is an infinite tensor product or not generally depends on the specific choice of that object. In particular, there may be infinite tensor products which are not Kolmogorov products. For example in $\CRing_+\op$, we may take the tensor product of abelian groups $\bigotimes_i R_i$ and equip it with a different multiplication than the tensor product of rings one (as long as it has the same unit).

\begin{proposition}
	Every Kolmogorov product is a categorical product in $\cC_\detc$.	
\end{proposition}

\begin{proof}
	For $(X_i)_{i \in J}$ a family of objects with Kolmogorov product $X_J$ and finite marginalizations $\pi_F : X_J \to X_F$, we first show that if we have a compatible family of deterministic morphisms $g_F : A \to X_F$, then also the induced $g_J : A \to X_J$ is deterministic. Drawing the Kolmogorov product $X_J$ as a double wire, we need to prove the determinism equation
	\[
		\tikzfig{gJ_det}
	\]
	By \Cref{two_infproducts}, the codomain object $X_J \otimes X_J$ is itself an infinite tensor product, and it is therefore enough to prove
	\[
		\tikzfig{gJ_det2}
	\]
	where we have assumed without loss of generality that the two finite marginalization maps are the same, which we can by replacing the corresponding finite subsets by their union. Using the determinism assumption for $\pi_F$ on the right as well as the assumption that $g_F = \pi_F \circ g_J$ is deterministic implies the claim.

	Now since every finite tensor product $X_F$ is a categorical product in $\cC_\detc$, we can use the fact that in every category,
    	\[
		\lim_{F \subseteq J \text{ finite}} \: \prod_{i \in F} X_i \: \cong \: \prod_{i \in J} X_i,
	\]
	and this is what was to be shown.
\end{proof}

In particular, this proves that the comonoid structure on a Kolmogorov product is determined uniquely: $\cop{\bigotimes_i X_i}$ has to be the diagonal map in $\cC_{\rm det}$, and this condition determines $\bigotimes_i X_i$ up to \emph{deterministic} isomorphism.

\todo[inline]{Add an example where $\cC_{\rm det}$ has infinite products, but $\cC$ lacks infinite tensor products? Like sets and multivalued functions?}

\todo[inline]{T: sounds good to me, can you put it in?}

The notion of Kolmogorov product describes a notion of infinite collection of random variables where all dependence must be encoded in the finite subsets of the variables. We capture this in the following definition.

\begin{definition}
	We say that a morphism $p_J : A \to \bigotimes_{i \in J} X_i$ \emph{displays the conditional independence} $\perp_{i \in J}  X_i \:|\: A$ if each finite marginalization $p_F : A \to \bigotimes_{i \in F}X_i$ displays the conditional independence $\perp_{i \in F} X_i \:|\: A$.
\end{definition}

\section{The zero--one laws}

\begin{theorem}
    \label{thm:kolmog}
    Suppose we are given $p: A \to T \tensor \bigotimes_{i \in J} X_i$ satisfying the following conditions:
    \begin{enumerate}
        \item $T$ is a deterministic function of $\bigotimes_{i \in J} X_i$.
        \item For each finite subset $F \subset J$, the marginal $A \to T \tensor \bigotimes_{i \in F}X_i$ displays the independence of $T$ and $\bigotimes_{i \in F} X_i$.
    \end{enumerate}
    Then the marginal $A \to T$ is deterministic as well.
\end{theorem}

This theorem is a consequence of two lemmas.
\begin{lemma}[The infinite independence lemma]
    Suppose $p: A \to \bigotimes X_i$ exhibits the independence of $\{X_i\}$.
    Then for each $j$, the map $A \to X_j \tensor \bigotimes_{i \neq j}X_i$ exhibits the independence $X_j \bot \tensor_{i\neq j} X_i \mid \mid A$.
\end{lemma}
\begin{proof}[Sketch of proof]
    To prove this, we must compare two maps $A \to X_j \tensor \bigotimes_{i\neq j} X_i \cong \bigotimes_i X_i$.
    To show that these maps are equal, it suffices to show that all finite marginalizations of them are equal.
    But this is an immediate consequence of the assumption.
\end{proof}


\begin{lemma}[The determinism lemma]
    Suppose $p: A \to T \tensor X$ is such that $T$ is a deterministic function of $X$, and $p$ exhibits the independence $T \bot X \mid \mid A$.
    Then the marginal $A \to T$ is deterministic.
\end{lemma}
\begin{proof}
The proof is a string diagram chase:
First, the assumption that $T$ is a deterministic function of $X$ means precisely that we can find $f$ deterministic so that.

%\tikzfig{determinismlemma1}.

Now we can use the independence of $X$ and $T$, and the axioms, to rewrite this as

%\tikzfig{determinismlemma2}

Applying the map $1_T \tensor f$ to the second and fourth diagram now gives an equality

%\tikzfig{determinismlemma3}

Where we have also applied the determinism of $f$.
This is precisely the statement that $A \to T \tensor X \to X \labelto{f} T$ is deterministic.

Lastly, we use the equality 

%\tikzfig{determinismlemma4}

To see that the marginal $A \to T \tensor X \to T$ is deterministic as desired.

\end{proof}


\begin{proof}[Proof of the theorem]
    It's clear from the definition that $A \to T \tensor \bigotimes X_i$ exhibits the independence of $\{T, X_1, \dots\}$.
    By the infinite independence lemma, this means that it also exhibits the independence $T \bot \bigotimes X_i \mid \mid A$.
    Now the determinism lemma implies exactly what we want, that the marginal $A \to T$ is deterministic.
\end{proof}

\begin{corollary}[Kolmogorov zero to one law]
    Suppose $\Omega$ is a measure space with a probability measure $P$, $\{f_i: \Omega \to X_i\}$ is a collection of independent random variables,
    and $T \subset \Omega$ is a subset in the $\sigma$-algebra generated by the $f_i$, such that $1_T$ is independent of any finite subset of the $f_i$.
    Then $P(T)$ is $0$ or $1$.
\end{corollary}
\begin{proof}
    Consider the composite $I \labelto{P} \Omega \labelto{1_T, \{f_i\}} \{0,1\} \tensor \bigotimes_i X_i$.
    Then independence of the $f_i$ means precisely that this map exhibits the independence of $\{X_i\}$ given $I$ in the above sense.
    Moreover, $T(\omega)$ is determined by the values of $f_i(\omega)$, so $T$ factors as a map $\prod_i X_i \to \{0,1\}$, and
    this map is measurable.
    Hence we can apply the theorem, and conclude that the map $I \labelto{P} \Omega \labelto{T} \{0,1\}$ is deterministic - but this means it's just a constant map,
    which precisely means that $T$ is true or false with probability $1$.
\end{proof}

We can also prove a version of the Hewitt-Savage theorem
\begin{definition}
    Let $\alpha: J \to J$ be any map.
    Then it induces a map 
    \[\hat{\alpha}: \bigotimes_{i\in J}X_i \to \bigotimes_{i \in J}X_{\alpha(i)},\]
    which we may suggestively write as $(x_i)_{i\in J} \mapsto (x_{\alpha(i)})_{i\in J}$.
    
    It is clear that this map is always deterministic. Moreover, if $\alpha$ is a bijection, this map is an isomorphism.
\end{definition}

\begin{theorem}[Hewitt-Savage for Markov categories]
    \label{thm:hewsav}
    Let $\cC$ be a causal Markov category with infinite tensor products. Let $A \to \cC$ be a map in $\cC$, $J$ an infinite set, and let $f: \bigotimes_{i \in J}X \to T$ be a deterministic map.
    Suppose for each finite permutation $\sigma: J \to J$, we have $f \circ \hat{\sigma} = f$ - in other words, $f$ is independent of finite permutations of the inputs.
    Then the composite $A \to \tensor_J A \to \tensor_J X \to T$ is deterministic.
\end{theorem}
Again, we factor part of the proof into a lemma.
\begin{lemma}
    Let $\cC$ be a causal markov category, and let $f,g: B \to X$ be maps, $f$ deterministic.
    Let $p: A \to B$ be any map.
    Suppose the composites 
    \[A \to B \to B \tensor B \labelto{f \tensor g} X \tensor X\]
    and
    \[A \to B \to B \tensor B \labelto{f \tensor f} X \tensor X\]
    agree. Then $f = g$ $p$-almost everywhere.
\end{lemma}
\begin{proof}[Sketch of sketch of proof]
    Consider the map
    $A \labelto{p} B \to B \tensor B \labelto{f \tensor g} X \tensor X \labelto{copy_{X \tensor X}} X \tensor X \tensor X \tensor X$.
    Whether we marginalize on the first or second factor of $X$, we get equal morphisms, because we may replace the first part of this with
    $A \to B \labelto{f} B \to X \tensor X \tensor X \tensor X$ by using the assumptions and determinism.

    Then applying causality, with $f = p$, $g= (f \tensor g) \circ copy_B$, $h_1 = (del_X \tensor 1_X)$, $h_2  = (1_X \tensor del_X)$.
    Marginalizing the resulting (equal) diagrams, one of them is 
    \[A \to B \to B \tensor B \labelto{f, 1_B} B \tensor B,\]
    one of them is 
    \[A \to B \to B \tensor B \labelto{g,1_B} B \tensor B,\]
    and this is precisely what must hold for them to agree $p$-a.e.
\end{proof}

\begin{proof}[Proof of theorem]
    Consider the map 

    \[e: A \to \tensor_J A \labelto{\tensor p} (\tensor_J X) \to (\tensor_J X) \tensor (\tensor_J X) \to T \tensor (\tensor_J X)\]

    Denote by $e_\sigma$ the composite of $e$ with $1_T \tensor \hat{\sigma}$
    Note that if $\sigma$ is a finite permutation, $e_\sigma = e$. To see this, note that we may add a $\hat{\sigma}$ before the $f$ as well, by assumption.
    Now we can use the determinism of the $\hat{\sigma}$s, and the clear fact that 
    \[A \to \tensor A \to \tensor B \labelto{\hat{\sigma}} \tensor B = A \to \tensor A \to \tensor B\]

    Now let $\sigma$ instead be an injection. Then for each finite subset $F \subset J$, we can find a finite permutation of $J$, $\sigma_0$, such that $\sigma|_F = \sigma_0|_F$.
    Then on this marginal, $e_\sigma = e_{\sigma_0} = e$, hence $e_\sigma = e$ in general.

    Let $J = J_0 \sqcup J_1$ be a decomposition of $J$ into two disjoint subsets with the cardinality of $J$ (here we use $J$ infinite).
    Let $\tau_0, \tau_1: J \to J$ be injections with image $J_0,J_1$ respectively.

    Now consider the map $a_\sigma: A \to T \tensor T$ given by composing $e_\sigma$ with $t$.
    By the above, they all agree.
    By the lemma, we see that the maps $\tensor_J X \labelto{\hat{\sigma}} \tensor_J X \to T$ are all $\tensor_J p$ a.e. equal.
    Applying this to $\tau_0,\tau_1$, we get that the map $A \to T$ must be deterministic.
\end{proof}
In the last step, the idea is that precomposing with $\hat{\tau_0}$ and $\hat{\tau_1}$ is essentially picking out two disjoint subsets and using them to determine $T$.
Since all the variables are independent, these two are clearly independent, but also equal (by the preceding argument). Hence they are deterministic, but they also agree with the original map.

\section{Examples}
As noted, theorems \cref{thm:kolmog} and \cref{thm:hewsav}, when intepreted in the Markov category $\mathsf{Stoch}$, recover the classical Kolmogorov and Hewitt-Savage zero to one laws.
\improve{Maybe move those examples to here? Also figure out where to put proof that $Kl(H)$ has infinite tensor product and Kolmogorov extension. (And introduce $H$)}

By intepreting them in other Markov categories, we obtain other nontrivial (although hardly particularly deep) statements:

\begin{corollary}
    Let $\{X_i\}_{i \in J}$ be a family of topological spaces, $Y$ a Hausdorff space, and let $f: \prod_i X_i \to Y$ be a continuous function
    which is independent of any finite prefix of the input.
    Then the map $f$ is constant.
\end{corollary}
\begin{corollary}
    Let $X, Y$ be topological spaces, $Y$ Hausdorff, and let $f: \prod_{i\in J} X \to Y$ be a continuous function which is independent of any finite permutation of the input.
    Then $f$ is constant.
\end{corollary}
\begin{proof}
    The theorems follow from \cref{thm:kolmog} and \cref{thm:hewsav}, applied to the Kleisli category $Kl(H)$ of the hyperspace monad,
    using the maps $* \to X_i$ and $* \to X$ given by the closed subsets $X_i \subseteq X_i, X \subset X$.
    The conclusion is that the closure of the image, call it $A$, has the following property: $\overline{\{(a,a) \mid a \in A\}} = A \times A$ as subsets of $Y \times Y$.
    But since the diagonal is closed in $Y$, this clearly implies that $A$ must be a singleton.
\end{proof}


\section{Speculation}
\todo[inline]{not sure if this section is any good}
It would good to have a way of using this theorem without assuming that the category in question has these infinite tensor products.
We can always make sense of the notion of an infinite independent collection of random variables (just a collection of maps $I \to X_i$).
Of course to talk about an outcome or a function which depends on the outcome of all the $X_i$, we need a map $\prod_i X_i \to T$ of some type.
But since this map is deterministic, we may want to talk about it without admitting it to our category of Markov kernels.

As an example, the probability theory of finite or countable sets is much simpler than the probability theory of infinite sets, since the whole edifice of $\sigma$-algebras can be done away with.
This is a reason to prefer $\sf{FinStoch}$ over the full $\sf{Stoch}$.
In this case, it would be desirable to be able to apply this theorem, given a collection of finite variables $I \to X_i$ and a function $\prod_i X_i \to Y$
(which may be required to satisfy some condition like continuity in the product topology), without begin required to invent the larger category $\sf{Stoch}$.

For instance, maybe there is a Markov category with objects the profinite sets, which can be obtained from $\sf{FinStoch}$ by some formal construction (maybe even simply as the pro-category), which admits infinite tensor products, and where one can profitably interpret this theorem.


\bibliographystyle{plain}
\bibliography{categorical-zero-one}

\end{document}
