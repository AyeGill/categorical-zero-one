\documentclass[11pt]{article}
\usepackage{eigilscmds}
\usepackage{tikzit}
\usepackage{geometry}
\usepackage{enumitem}
\usepackage{csquotes}
\input{comonoids.tikzdefs}
\input{comonoids.tikzstyles}

% draft packages
%\usepackage{todonotes}
%\usepackage{showkeys}

\author{Tobias Fritz and Eigil Fjeldgren Rischel}
\title{Response to Referees:\\ The zero--one laws of Kolmogorov and Hewitt--Savage\\ in categorical probability}
\date{\today}

\renewcommand{\sf}{\mathsf}
\DeclareMathOperator{\del}{del}
\begin{document}
\maketitle

\begin{abstract}
	We would like to thank both referees for their useful suggestions. There are some items that we would like to respond to, in particular from the first referee's report.
\end{abstract}

\section*{Referee 1}

We have been quite surprised by the length and the amount of detail of this report, and it is exciting that the referee has invested so much work into it. The referee's suggestion of putting more emphasis on our infinite tensor products and Kolmogorov products has been helpful for us, and we have shifted the emphasis a bit in that direction.

At the same time, we strongly disagree with the referee on essentially all other counts. In particular, the details below will explain that \textbf{all ``errors'' pointed out in the report are the referee's errors, not ours}. We start by commenting on the summary report and then move on to the material in the attached PDF.

\begin{enumerate}[label=(\alph*).]
	\item The referee writes:
		\begin{displayquote}
			Your definition of the infinite tensor product is not correct - it does not capture the property characterizing the Kolmogorov extension theorem.  This can be corrected easily enough.
		\end{displayquote}
		This seems to claim that there is an error in our paper.
		\begin{displayquote}	
		However, that correction does require additional material proving that the Kolmogorov extension theorem really is equivalent to the existence of infinite tensor products for the category of standard Borel spaces. The difficulty arises because it is necessary to show that the Kleisi category construction, which is the framework in which  the Kolmogorov extension theorem is given, is a special case of the more general situation in a Markov semicartesian symmetric monoidal category where we have an inclusion of a subcategory (specified by the  deterministic morphisms) into a category which consist of deterministic and nondeterministic mappings.
		\end{displayquote}
		Now the error seems to have become a mere gap. Regardless, there is no such error or gap: the connection with the Kolmogorov extension theorem is explained in Remark 3.5 and Example 3.6. In order to see that this recovers the extension theorem, it is not necessary to construct $\BorelStoch$ as the Kleisli category of the Giry monad (although this is certainly one way to do it); the concrete construction of $\BorelStoch$ in terms of Markov kernels and the Chapman--Kolmogorov formula is perfectly enough, and we have explained why countable tensor products in our sense exist in $\BorelStoch$, and it is obvious that this specializes to the Kolmogorov extension theorem for countable products of standard Borel spaces.

		In fact, the referee already seems to agree with this: the relevant diagram is exactly the referee's one on the second half of p.6 (in the report). The ``new terminology'' that the referee proposes surely does not have any bearing on whether the universal property described by the diagram holds in $\BorelStoch$, right?
\end{enumerate}

On the comments in the PDF:

\begin{enumerate}[resume,label=(\alph*).]
	\item The referee's proposed new formulation of Definition 3.1 is along the lines of our Definition 4.1.\footnote{With the only difference between the preservation by the functors $-\otimes Y$, which we get to in the next item.} In both definitions, the Kolmogorov product is the limit of the finite products $\bigotimes_{i \in F} X_i$ in both $\cC$ and in $\cC_{\det}$. So it is not clear to us what the referee's point is. 

		Perhaps the referee's suggestion is to replace Definition 3.1 by Definition 4.1. We do not want to do this, since the latter only applies in Markov categories, while the former more generally makes sense in semicartesian monoidal categories.
		\begin{displayquote}
			and hence every ``infinite tensor product'' is Kolmogorov.
		\end{displayquote}
		This is true in categories like $\Stoch$ and $\BorelStoch$, but not in general. We had already explained this point after Definition 4.1. In conclusion, the distinction between infinite tensor products (in semicartesian monoidal categories, Section 3) and Kolmogorov products (in Markov categories, Section 4) is important and should not be swept under the rug.
	\item Quoting from p.7,
		\begin{displayquote}
			I did not incorporate the property of preservation of lim $\mathcal{D}$ by the endofunctor
			$-\otimes Y : \cC_{\det} \to \cC_{\det}$ because that condition is a global property in the sense
		that $Y$ is an arbitrary element in your space which may have no relationship
		to your fixed family $(X_j)_{j\in J}$ for which you are trying to find an extension.
		\end{displayquote}
		Assuming that ``space'' means category, we agree that there is no relation between $Y$ and the family $(X_i)$, but we do not see why and how this would be an argument against our preservation condition. As the referee will hopefully have noticed, this preservation condition is an important property which we use throughout the paper; in particular, the preservation property is what guarantees the coherence property described in Remark 3.3.

		Perhaps the referee's point is that this preservation condition never appears in the context of Kolmogorov's extension theorem. This is true, and the reason is simple: it \emph{automatically holds} in $\BorelStoch$, as we have explained in Example 3.6. Hence there is no reason to state it explicitly in the standard probability context.
	\item On p.8, the referee again seems to conclude the incorrectness of a statement from their lack of understanding of our arguments:
		\begin{displayquote}
		(2) Page 17, the last sentence. The statement
			\begin{displayquote}
			Moreover, $\hat{\sigma}$ is deterministic by Proposition 4.3.
			\end{displayquote}
		is not correct.
		\end{displayquote}
		The comment that follows:
			\begin{displayquote}
				then please cite a reference	
			\end{displayquote}
		indeed makes it clear that the referee does not claim to have found a mistake. In fact, the reason that $\hat{\sigma}$ is deterministic is by a trivial application of the universal property of the Kolmogorov product as a limit in $\cC_{\det}$, based precisely on the facts that the referee mentions: both the composite and the $\pi_F$'s are deterministic, and hence the factorization exists uniquely.
	\item Again p.8:
		\begin{displayquote}
			(3) Corollary 6.1. (Kolmogorov zero-one law) Your assumption, which says that
			$T \subseteq \Omega$, is incorrect, because if $T$ is just a subset of $\Omega$, then I can just choose
			it to be any nonmeasurable set of $\Omega$.
		\end{displayquote}
		It is a mystery to us how it is possible to read over half of a sentence: the relevant measurability assumption was stated in clear and explicit terms directly after $T \subseteq \Omega$.
		\begin{displayquote}
			If you
			meant to say that $T$ is a measurable set in $\Omega$, the whole problem statement is an
			elementary result in measure theory, and certainly not the Kolmogorov zero-one
			law.
		\end{displayquote}
		If $T$ was an \emph{arbitrary} measurable set in $\Omega$, then the statement would quite obviously be false. In any case, our statement of the Kolmogorov zero--one law has been and is exactly the classical one, and it is so classical that it should not require a reference.
	\item p.9, item (2): Even if most categorical foundations in algebra and geometry required monoidal closure (which is obviously not the case, take e.g.~abelian categories), we do not see how this would be an argument for monoidal closure in the context of probability theory.
	\item p.9, item (4): Standard usage of the term ``diagram'' in category theory does not have any association with graph homomorphisms. In standard parlance as we use it, ``diagram'' is just any functor from a small category into the category under consideration.\footnote{See e.g.~\href{https://ncatlab.org/nlab/show/diagram\#CategoryShapedDiagramFunctorially}{https://ncatlab.org/nlab/show/diagram\#CategoryShapedDiagramFunctorially}.}
	\item p.11:
		\begin{displayquote}
			\textbf{Theorem 4.1.} The weak limit of the functor $\mathcal{D}' : \mathscr{J} \to \Stoch$ exists. Moreover, the weak limit of the functor $\mathcal{D}'$ is never a limit in $\Stoch$ except for trivial cases.
		\end{displayquote}
		It is not clear to us what ``the weak limit'' means, since weak limits are rarely unique (up to unique iso), and in particular uniqueness does not hold in the case under consideration. That being said, we agree that the referee's construction indeed produces a weak limit that is not a limit. But the fact that that particular weak limit is not a limit says nothing about whether the functor has a limit or not.
\end{enumerate}

\section*{Referee 2}

We thank the referee for a number of suggestions which have led to an improvement of the presentation. It is good to know for us that the paper is not so easy to follow at times, and we have invested some work into the exposition with the referee's feedback in mind. 

%\bibliographystyle{plain}
%\bibliography{categorical-zero-one}

\end{document}




